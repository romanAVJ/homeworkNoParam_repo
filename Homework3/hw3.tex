\documentclass[]{article}
\usepackage{lmodern}
\usepackage{amssymb,amsmath}
\usepackage{ifxetex,ifluatex}
\usepackage{fixltx2e} % provides \textsubscript
\ifnum 0\ifxetex 1\fi\ifluatex 1\fi=0 % if pdftex
  \usepackage[T1]{fontenc}
  \usepackage[utf8]{inputenc}
\else % if luatex or xelatex
  \ifxetex
    \usepackage{mathspec}
  \else
    \usepackage{fontspec}
  \fi
  \defaultfontfeatures{Ligatures=TeX,Scale=MatchLowercase}
\fi
% use upquote if available, for straight quotes in verbatim environments
\IfFileExists{upquote.sty}{\usepackage{upquote}}{}
% use microtype if available
\IfFileExists{microtype.sty}{%
\usepackage{microtype}
\UseMicrotypeSet[protrusion]{basicmath} % disable protrusion for tt fonts
}{}
\usepackage[margin=1in]{geometry}
\usepackage{hyperref}
\hypersetup{unicode=true,
            pdftitle={hw3},
            pdfborder={0 0 0},
            breaklinks=true}
\urlstyle{same}  % don't use monospace font for urls
\usepackage{color}
\usepackage{fancyvrb}
\newcommand{\VerbBar}{|}
\newcommand{\VERB}{\Verb[commandchars=\\\{\}]}
\DefineVerbatimEnvironment{Highlighting}{Verbatim}{commandchars=\\\{\}}
% Add ',fontsize=\small' for more characters per line
\usepackage{framed}
\definecolor{shadecolor}{RGB}{248,248,248}
\newenvironment{Shaded}{\begin{snugshade}}{\end{snugshade}}
\newcommand{\AlertTok}[1]{\textcolor[rgb]{0.94,0.16,0.16}{#1}}
\newcommand{\AnnotationTok}[1]{\textcolor[rgb]{0.56,0.35,0.01}{\textbf{\textit{#1}}}}
\newcommand{\AttributeTok}[1]{\textcolor[rgb]{0.77,0.63,0.00}{#1}}
\newcommand{\BaseNTok}[1]{\textcolor[rgb]{0.00,0.00,0.81}{#1}}
\newcommand{\BuiltInTok}[1]{#1}
\newcommand{\CharTok}[1]{\textcolor[rgb]{0.31,0.60,0.02}{#1}}
\newcommand{\CommentTok}[1]{\textcolor[rgb]{0.56,0.35,0.01}{\textit{#1}}}
\newcommand{\CommentVarTok}[1]{\textcolor[rgb]{0.56,0.35,0.01}{\textbf{\textit{#1}}}}
\newcommand{\ConstantTok}[1]{\textcolor[rgb]{0.00,0.00,0.00}{#1}}
\newcommand{\ControlFlowTok}[1]{\textcolor[rgb]{0.13,0.29,0.53}{\textbf{#1}}}
\newcommand{\DataTypeTok}[1]{\textcolor[rgb]{0.13,0.29,0.53}{#1}}
\newcommand{\DecValTok}[1]{\textcolor[rgb]{0.00,0.00,0.81}{#1}}
\newcommand{\DocumentationTok}[1]{\textcolor[rgb]{0.56,0.35,0.01}{\textbf{\textit{#1}}}}
\newcommand{\ErrorTok}[1]{\textcolor[rgb]{0.64,0.00,0.00}{\textbf{#1}}}
\newcommand{\ExtensionTok}[1]{#1}
\newcommand{\FloatTok}[1]{\textcolor[rgb]{0.00,0.00,0.81}{#1}}
\newcommand{\FunctionTok}[1]{\textcolor[rgb]{0.00,0.00,0.00}{#1}}
\newcommand{\ImportTok}[1]{#1}
\newcommand{\InformationTok}[1]{\textcolor[rgb]{0.56,0.35,0.01}{\textbf{\textit{#1}}}}
\newcommand{\KeywordTok}[1]{\textcolor[rgb]{0.13,0.29,0.53}{\textbf{#1}}}
\newcommand{\NormalTok}[1]{#1}
\newcommand{\OperatorTok}[1]{\textcolor[rgb]{0.81,0.36,0.00}{\textbf{#1}}}
\newcommand{\OtherTok}[1]{\textcolor[rgb]{0.56,0.35,0.01}{#1}}
\newcommand{\PreprocessorTok}[1]{\textcolor[rgb]{0.56,0.35,0.01}{\textit{#1}}}
\newcommand{\RegionMarkerTok}[1]{#1}
\newcommand{\SpecialCharTok}[1]{\textcolor[rgb]{0.00,0.00,0.00}{#1}}
\newcommand{\SpecialStringTok}[1]{\textcolor[rgb]{0.31,0.60,0.02}{#1}}
\newcommand{\StringTok}[1]{\textcolor[rgb]{0.31,0.60,0.02}{#1}}
\newcommand{\VariableTok}[1]{\textcolor[rgb]{0.00,0.00,0.00}{#1}}
\newcommand{\VerbatimStringTok}[1]{\textcolor[rgb]{0.31,0.60,0.02}{#1}}
\newcommand{\WarningTok}[1]{\textcolor[rgb]{0.56,0.35,0.01}{\textbf{\textit{#1}}}}
\usepackage{graphicx,grffile}
\makeatletter
\def\maxwidth{\ifdim\Gin@nat@width>\linewidth\linewidth\else\Gin@nat@width\fi}
\def\maxheight{\ifdim\Gin@nat@height>\textheight\textheight\else\Gin@nat@height\fi}
\makeatother
% Scale images if necessary, so that they will not overflow the page
% margins by default, and it is still possible to overwrite the defaults
% using explicit options in \includegraphics[width, height, ...]{}
\setkeys{Gin}{width=\maxwidth,height=\maxheight,keepaspectratio}
\IfFileExists{parskip.sty}{%
\usepackage{parskip}
}{% else
\setlength{\parindent}{0pt}
\setlength{\parskip}{6pt plus 2pt minus 1pt}
}
\setlength{\emergencystretch}{3em}  % prevent overfull lines
\providecommand{\tightlist}{%
  \setlength{\itemsep}{0pt}\setlength{\parskip}{0pt}}
\setcounter{secnumdepth}{0}
% Redefines (sub)paragraphs to behave more like sections
\ifx\paragraph\undefined\else
\let\oldparagraph\paragraph
\renewcommand{\paragraph}[1]{\oldparagraph{#1}\mbox{}}
\fi
\ifx\subparagraph\undefined\else
\let\oldsubparagraph\subparagraph
\renewcommand{\subparagraph}[1]{\oldsubparagraph{#1}\mbox{}}
\fi

%%% Use protect on footnotes to avoid problems with footnotes in titles
\let\rmarkdownfootnote\footnote%
\def\footnote{\protect\rmarkdownfootnote}

%%% Change title format to be more compact
\usepackage{titling}

% Create subtitle command for use in maketitle
\providecommand{\subtitle}[1]{
  \posttitle{
    \begin{center}\large#1\end{center}
    }
}

\setlength{\droptitle}{-2em}

  \title{hw3}
    \pretitle{\vspace{\droptitle}\centering\huge}
  \posttitle{\par}
    \author{}
    \preauthor{}\postauthor{}
      \predate{\centering\large\emph}
  \postdate{\par}
    \date{30/3/2020}


\begin{document}
\maketitle

\begin{Shaded}
\begin{Highlighting}[]
\KeywordTok{set.seed}\NormalTok{(}\DecValTok{20200316}\NormalTok{) }\CommentTok{#created at date}

\CommentTok{# assigning number of excercises }
\NormalTok{(people_3quest <-}\StringTok{ }\KeywordTok{sample}\NormalTok{(}\DataTypeTok{x =} \KeywordTok{c}\NormalTok{(}\StringTok{"Luis"}\NormalTok{,}\StringTok{"Roman"}\NormalTok{,}\StringTok{"Sant"}\NormalTok{,}\StringTok{"Sof"}\NormalTok{),}\DataTypeTok{replace =}\NormalTok{ F, }\DataTypeTok{size =} \DecValTok{2}\NormalTok{)) }\CommentTok{#people with 3 questions, by alphabetic order}
\end{Highlighting}
\end{Shaded}

\begin{verbatim}
## [1] "Sof"  "Luis"
\end{verbatim}

\begin{Shaded}
\begin{Highlighting}[]
\CommentTok{# assinging excersises}
\NormalTok{number_ex <-}\StringTok{ }\DecValTok{1}\OperatorTok{:}\DecValTok{10} \CommentTok{#excercises}
\KeywordTok{cat}\NormalTok{(}\StringTok{"}\CharTok{\textbackslash{}n}\StringTok{Ejercicios Luis: "}\NormalTok{)}
\end{Highlighting}
\end{Shaded}

\begin{verbatim}
## 
## Ejercicios Luis:
\end{verbatim}

\begin{Shaded}
\begin{Highlighting}[]
\NormalTok{(ex_luis <-}\StringTok{ }\KeywordTok{sample}\NormalTok{(}\DataTypeTok{x =}\NormalTok{ number_ex, }\DataTypeTok{replace =}\NormalTok{ F, }\DataTypeTok{size =} \DecValTok{3}\NormalTok{))}
\end{Highlighting}
\end{Shaded}

\begin{verbatim}
## [1] 2 3 6
\end{verbatim}

\begin{Shaded}
\begin{Highlighting}[]
\NormalTok{number_ex <-}\StringTok{ }\NormalTok{number_ex[}\OperatorTok{!}\StringTok{ }\NormalTok{number_ex }\OperatorTok\StringTok{ }\NormalTok{ex_luis] }\CommentTok{#removing questions }

\KeywordTok{cat}\NormalTok{(}\StringTok{"}\CharTok{\textbackslash{}n}\StringTok{Ejercicios Roman: "}\NormalTok{)}
\end{Highlighting}
\end{Shaded}

\begin{verbatim}
## 
## Ejercicios Roman:
\end{verbatim}

\begin{Shaded}
\begin{Highlighting}[]
\NormalTok{(ex_roman <-}\StringTok{ }\KeywordTok{sample}\NormalTok{(}\DataTypeTok{x =}\NormalTok{ number_ex, }\DataTypeTok{replace =}\NormalTok{ F, }\DataTypeTok{size =} \DecValTok{2}\NormalTok{))}
\end{Highlighting}
\end{Shaded}

\begin{verbatim}
## [1] 5 4
\end{verbatim}

\begin{Shaded}
\begin{Highlighting}[]
\NormalTok{number_ex <-}\StringTok{ }\NormalTok{number_ex[}\OperatorTok{!}\StringTok{ }\NormalTok{number_ex }\OperatorTok\StringTok{ }\NormalTok{ex_roman] }\CommentTok{#removing questions }

\KeywordTok{cat}\NormalTok{(}\StringTok{"}\CharTok{\textbackslash{}n}\StringTok{Ejercicios Sant: "}\NormalTok{)}
\end{Highlighting}
\end{Shaded}

\begin{verbatim}
## 
## Ejercicios Sant:
\end{verbatim}

\begin{Shaded}
\begin{Highlighting}[]
\NormalTok{(ex_sant <-}\StringTok{ }\KeywordTok{sample}\NormalTok{(}\DataTypeTok{x =}\NormalTok{ number_ex, }\DataTypeTok{replace =}\NormalTok{ F, }\DataTypeTok{size =} \DecValTok{2}\NormalTok{))}
\end{Highlighting}
\end{Shaded}

\begin{verbatim}
## [1] 10  8
\end{verbatim}

\begin{Shaded}
\begin{Highlighting}[]
\NormalTok{number_ex <-}\StringTok{ }\NormalTok{number_ex[}\OperatorTok{!}\StringTok{ }\NormalTok{number_ex }\OperatorTok\StringTok{ }\NormalTok{ex_sant] }\CommentTok{#removing questions }

\KeywordTok{cat}\NormalTok{(}\StringTok{"}\CharTok{\textbackslash{}n}\StringTok{Ejercicios Sof: "}\NormalTok{)}
\end{Highlighting}
\end{Shaded}

\begin{verbatim}
## 
## Ejercicios Sof:
\end{verbatim}

\begin{Shaded}
\begin{Highlighting}[]
\NormalTok{(ex_sof <-}\StringTok{ }\NormalTok{number_ex)}
\end{Highlighting}
\end{Shaded}

\begin{verbatim}
## [1] 1 7 9
\end{verbatim}

\hypertarget{ejercicio-4}{%
\section{Ejercicio 4}\label{ejercicio-4}}

\hypertarget{sea-t_i-fracx_i-barxs.-mostrar-que-t_i-fracn-1sqrtn.}{%
\subsection{\texorpdfstring{Sea \(t_{i} = \frac{x_{i}-\bar{x}}{s}\).
Mostrar que
\(|t_{i}| < \frac{n-1}{\sqrt{n}}\).}{Sea t\_\{i\} = \textbackslash frac\{x\_\{i\}-\textbackslash bar\{x\}\}\{s\}. Mostrar que \textbar t\_\{i\}\textbar{} \textless{} \textbackslash frac\{n-1\}\{\textbackslash sqrt\{n\}\}.}}\label{sea-t_i-fracx_i-barxs.-mostrar-que-t_i-fracn-1sqrtn.}}

\emph{Demostración}

\begin{enumerate}
\def\labelenumi{\arabic{enumi}.}
\tightlist
\item
  Supongamos tamaño de muestra \(n\) arbitraria pero fija.
\item
  Sin pérdida de generalidad, supongamos que
  \(x_{n} \geq x_{i}, \quad \forall i \in \{1,\dots,n-1\}\).
\item
  Supongamos que \(\bar{x}_{n-1} = 0\) sin pérdida de generalidad.
\end{enumerate}

Por un lado tenemos que
\[ \bar{x}_{n} = \frac{x_{n} + \sum_{i = 1}^{n}{x_{i}}}{n} = \frac{x_{n}}{n}.\]

Por otro lado se tiene

\[s_{n}^{2} = \frac{\sum_{i=1}^{n}{(x_{i} - \bar{x}_{n})^{2}}}{n-1} = \frac{\sum_{i=1}^{n-1}{(x_{i}^{2} - \frac{2}{n}x_{i}x_{n} + \frac{x_{n}^{2}}{n^{2}})} + (x_{n} - \frac{x_{n}}{n})^{2}}{n-1} = \frac{n-2}{n-1}s^{2}_{n-1} + \frac{1}{n^{2}}x_{n}^{2}.\]
Por simplicidad en el álgebra, tomemos
\(\beta = \frac{n-2}{n-1}s^{2}_{n-1}\) y \(\alpha = \frac{1}{n^{2}}.\)

Definamos \[g(\bar{x},s^{2}) = \frac{x_{n}-\bar{x}}{s^{2}}.\]

Por hipotesis y al tener \(n\) fija, tenemos que
\[t_{i} = \frac{x_{i}-\bar{x}_{n}}{s_{n}^{2}} \leq \frac{x_{n}-\bar{x}_{n}}{s^{2}_{n}} = g(\bar{x},s^{2}).\]

Pero al tener \(n\) fia se ve que
\[g(\bar{x},s^{2}) = \frac{x_{n}-\bar{x}_{n}}{s^{2}_{n}} = \frac{\frac{n-1}{n}x_{n}}{\sqrt{\frac{n-2}{n-1}s^{2}_{n-1} + \frac{1}{n^{2}}x_{n}^{2}}} = g(x_{n}).\]
Notemos que
\[\sup_{x_{n}}{g(x_{n})} = \lim_{x \to \infty }{g(x_{n})} = \frac{\frac{n-1}{n}}{\sqrt{\frac{1}{n^{2}}}} = \frac{n-1}{\sqrt{n}}\]

Ya que \(g\) no posee PCE, pero es monotonamente creciente después del
\(0\) y está acotada superiormente, por tanto su supremo existe y es el
límite. An+alogamente para el ínfimo, ya que es monótonamente
decreciente después del \(0\) y está acotada inferiormente. Por lo tanto
concluímos que \[ |t_{i}| < \frac{n-1}{\sqrt{n}}.\]

\emph{Específicamente}, vemos que si \(n \leq 10\) entonces
\(|t_{i}| < \frac{10-1}{\sqrt{10}} < \frac{10-1}{\sqrt{9}} = 3.\)

\hypertarget{ejercicio-5}{%
\section{Ejercicio 5}\label{ejercicio-5}}

\hypertarget{ejercicio-5a.-encontrar-el-riq-para-una-v.a.-normal.}{%
\subsection{Ejercicio 5a. Encontrar el RIQ para una v.a.
normal.}\label{ejercicio-5a.-encontrar-el-riq-para-una-v.a.-normal.}}

Para encontrar el \emph{rango intercuantíl} de una v.a.
\(X \sim \mathcal{N}(\mu,\sigma^{2})\), se ve que
\[ \mathbb{P}[x \leq \kappa_{0.25}] = 0.25 \iff \frac{\kappa_{0.25} - \mu}{\sigma} = -0.675 \iff \kappa_{0.25} = \mu - 0.675 \sigma.\]
Como la distribución de \(X\) es simétrica al rededor de \(\mu\), se
sigue que \(\kappa_{0.75} = \mu + 0.675\sigma.\)

Entonces tenemos que el
\(\text{RIQ}(X) = \kappa_{0.75} - \kappa{0.25} = 1.35 \sigma.\)

\hypertarget{ejercicio-5b.-determinar-la-constate-c-tal-que-el-riq-sea-un-estimador-consistente-de-sigma}{%
\subsection{\texorpdfstring{Ejercicio 5b. Determinar la constate \(c\)
tal que el RIQ sea un estimador consistente de
\(\sigma\)}{Ejercicio 5b. Determinar la constate c tal que el RIQ sea un estimador consistente de \textbackslash sigma}}\label{ejercicio-5b.-determinar-la-constate-c-tal-que-el-riq-sea-un-estimador-consistente-de-sigma}}

Si hacemos \(c = 1.35\), vemos que \(\text{RIQ}(X)/c = \sigma.\) Por lo
que de esta manera tendremos un estimador consistente para \(\sigma\).


\end{document}
